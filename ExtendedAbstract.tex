\documentclass[12pt]{article}
\usepackage[utf8]{inputenc}

\title{Project Extended Abstract}
\author{Conor Holden }

\usepackage{hyperref}
\hypersetup{
	colorlinks=true,
	linkcolor=blue,
	filecolor=magenta,      
	urlcolor=cyan,
}




\begin{document}
	
	\maketitle
		
		
		
\section{Introduction}

	The purposed of this project was to create a virtual instrument of the uilleann pipes. There are
	
	\begin{itemize}
		
		\item Intro
			\begin{itemize}
				\item Describe 
			\end{itemize}
		
		
			\begin{itemize}
				\item Physical Modelling
				\item VST, AU interfaces that allow you to connect with DAWs
				\item Juce \& iplug 2
			\end{itemize}
		
	\end{itemize}

\section{Analysis}

	\subsection{Physical Modelling Synthesis}
 	

	\subsection{Subtractive Synthesis}
	
	One of the potential ways to create a virtual uilleann pipes would be using subtractive synthesis. This involved using complex
	waveforms and subtracting frequencies of the sound using filters, envelopes and low frequency oscillator. Following the
	research on subtractive synthesis for the uillean pipes by T. Ó Póil et al and using a fully featured subtractive synthesiser,
	in this case Sylenth1 by LennarDigital, it is possible to create a very convincing sound. Given enough time and experience with
	this type of synthesis, it would be possible to create an almost indistinguishable sound.
	
	However, creating a subtractive synthesiser within the time of this project just to play uilleann pipes would be difficult and
	frankly, pointless. Especially when there are many fully featured synthesisers with years of support and updates available, 
	even some that are free, which would be easier to use and create a better sounding result.
	
	\subsection{Sampler Synthesiser}
	Arguably the most obvious and intuitive way to create a virtual instrument is to create a sampler instrument. This type will
	play samples or sound recordings on a key press. Some samplers, such as drum kits, are as simple as that. Each key will play
	a single sound once. For other instruments, it can be more complicated where notes consist of multiple samples. There
	might be samples for the beginning, the sustain and the end of the note, as well as different intensities when playing a note.
	
	After analysing the previous two methods and attempting the latter, it was eventually decided on the implement the
	uilleann pipes as a sampler. 
	
	
	\subsection{Technologies}
	The goal is to make instrument plugins that can be used in digital audio work stations, such as Abelton Live or FL Studio,
	and to do that, it needs to be in a certain format. The two most popular formats are VST and AU. Virtual Studio Technology or
	VST, is an interface developed by Steinberg for their DAW Cubase that allowed third part party instruments and effects to be 
	used with it. It has been licensed to other DAW developers and has become the most popular format. 
	Audio Units or AU is a format developed by Apple for Logic Pro and have since been implemented into MacOS and IOS for 
	other applications to use. 
	
	Both VST and AU allow the creation of plugins but you would have to do all the hard work yourself. Instead, a higher level
	framework would be used. A potential candidate was iPlug2, an open source framework that can compile to many different 
	formats including VST and AU. However, it is relatively new from 2018, it lacked the documentation and tutorials needed
	to use it for this project.
	
	The second framework found and chosen is JUCE. Similarly to iPlug2, it can be compiled to many different formats from a 
	single code base. What differs is that JUCE is much older and has extensive documentation on all the classes it provides
	for create audio applications and audio processing. JUCE also has a relatively big community that create tutorials and provide
	help for newcomers with questions. The only issues are that JUCE has separate code layout for audio applications and plugins
	and it can be difficult translating tutorials for applications to plugins. As well, JUCE 6 was released in June, 2020 and 
	some information is for the older versions.
	
		
\section{Design}
	The uilleann pipes will be implemented as a sampler instrument using the JUCE plugin format. 
	
	\subsection{Layout}
	
	\subparagraph{}		
		
		
\section {evaluation}
need more samples, artefacts, distortion
		
	
	
	
	
	
	\section{References}
	\begin{itemize}
		\item \href{http://mural.maynoothuniversity.ie/4106/}
		{Analysis tools and Results for a Subtractive Synthesis Modelling of the Irish Uilleann Pipes }
		
	\end{itemize}	
		
		
\end{document}