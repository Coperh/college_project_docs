\documentclass[12pt]{article}
\usepackage[utf8]{inputenc}

\title{Project Extended Abstract}
\author{Conor Holden }

\usepackage{biblatex}
\addbibresource{references.bib}


\begin{document}
	
	\maketitle
		
\section{Introduction}

	The purposed of this project was to create a virtual instrument of the uilleann pipes. This instrument would accept MIDI input and can be used with a modern digital audio workstation. The instrument would be in the VST plugin format that would allow it to 
	interface with many audio programs. In the end, it should sound reasonably good and would be able to play both the chanter 
	and drones of the uilleann pipes.
	
	At first, the project explored the various ways of generating sounds in virtual instruments. Specifically looking into physical modelling, subtractive synthesis and, sampling. These methods were examined in relation to the uilleann pipes and determined
	which one was the easiest to implement within the time of the project and which one created the best quality sound. After examining
	them and testing two, the sampler was determined to best option.
	
	To implement the sampler, a framework was needed. Using the software development kit for the VST plugin format would require
	making everything from scratch, which would have been difficult with limited knowledge. The framework chosen was
	JUCE. This provided the ability to create plugins for various different formats and stand-alone applications. It has an
	extensive set of classes and functions for audio programming and creating interfaces. Using JUCE, it was possible
	to create a reasonable good sounding instrument using samples from the uilleann pipes.	

\section{Analysis}

	
	\subsection{Subtractive Synthesis}
	
	One of the potential ways to create a virtual uilleann pipes would be using subtractive synthesis. This involved using complex
	waveforms and subtracting frequencies of the sound using filters, envelopes and low frequency oscillator. Following the
	research on subtractive synthesis for the uilleann pipes\cite{uilleannSubtractive} by T. Ó Póil et al and using a fully featured subtractive synthesiser,
	in this case Sylenth1 by LennarDigital, it is possible to create a very convincing sound. Given enough time and experience with
	this type of synthesis, it would be possible to create an almost indistinguishable sound.
	
	However, creating a subtractive synthesiser within the time of this project just to play uilleann pipes would be difficult and
	frankly, pointless. Especially when there are many fully featured synthesisers with years of support and updates available, 
	even some that are free, which would be easier to use and create a better sounding result.
	
	\subsection{Sampler Synthesiser}
	Arguably the most obvious and intuitive way to create a virtual instrument is to create a sampler instrument. This type will
	play samples or sound recordings on a key press. Some samplers, such as drum kits, are as simple as that. Each key will play
	a single sound once. For other instruments, it can be more complicated where notes consist of multiple samples. There
	might be samples for the beginning, the sustain and the end of the note, as well as different intensities when playing a note.
	
	After analysing the previous two methods and attempting the latter, it was eventually decided on the implement the
	uilleann pipes as a sampler. From research, it seems like other people who create virtual uilleann pipes and bag pipes in
	general also use this method such as Universal Piper.
	
	
	\subsection{Technologies}
	The goal is to make instrument plugins that can be used in digital audio work stations, such as Ableton Live or FL Studio,
	and to do that, it needs to be in a certain format. The two most popular formats are VST and AU. Virtual Studio Technology or
	VST, is an interface developed by Steinberg for their DAW Cubase that allowed third part party instruments and effects to be 
	used with it. It has been licensed to other DAW developers and has become the most popular format. 
	Audio Units or AU is a format developed by Apple for Logic Pro and have since been implemented into MacOS and IOS for 
	other applications to use. 
	
	Both VST and AU allow the creation of plugins but you would have to do all the hard work yourself. Instead, a higher level
	framework would be used. A potential candidate was iPlug2, an open source framework that can compile to many different 
	formats including VST and AU. However, it is relatively new from 2018, it lacked the documentation and tutorials needed
	to use it for this project.
	
	The second framework found and chosen is JUCE. Similarly to iPlug2, it can be compiled to many different formats from a 
	single code base. What differs is that JUCE is much older and has extensive documentation on all the classes it provides
	for create audio applications and audio processing. JUCE also has a relatively big community that create tutorials and provide
	help for newcomers with questions. The only issues are that JUCE has separate code layout for audio applications and plugins
	and it can be difficult translating tutorials for applications to plugins. As well, JUCE 6 was released in June, 2020 and 
	some information is for the older versions.
	
		
\section{Design}
	The uilleann pipes will be implemented as a sampler instrument as a JUCE plugin. The plugin will consist of two sound 
	generators corresponding to the uilleann pipes. The chanter, which plays the melody and can receive midi input and the
	drones, which will play a constant bass note.
	
	\subsection{Layout}
	There will only be 5 controls on the screen. There will be 3 volume sliders. Both the chanter and the drones will have their
	separate volume slider and there will be a third master slider that will control the overall volume of the instrument.
	The plugin will have an on screen keyboard. This will allow the user to play notes on the chanter by clicking on screen.
	Finally will be a button to toggle the drone. When enabled the drone will play and disable it will stop. The idea is that
	this option can be toggled on and off using the DAW's automation.
	
	\subsection{Creating Sound With Sampling}		
	For this project, there was no access to uilleann pipes. Usually when creating a sample, you would record high quality audio
	of the instrument. For the uilleann pipes, this would be the chanter and the three drones. However, the project was limited to
	find relatively low quality samples online.
	
	The drone is still fairly simple. It will loop a single clip of audio when enabled. An amplitude envelope will added to it so
	the audio does not just turn on and off instantly. This makes it sound slightly more natural and avoid popping noises when 
	the audio is suddenly stopped.
	
	The chanter is more complex. Ideally there would be a sample for every note or every type of note, but there was only one
	for some time. So the pitch of the sampled needed to be changed for every key. This is done by getting the frequency of 
	the sample and playing it proportionally faster or slower to make the frequency of a different notes. The same note in 
	different octaves, an octave higher has a frequency twice as high and an octave lower has a frequency half of the current note.
	This means that the sample can either be played at twice the speed or half the speed to achieve these notes. For other notes,
	it involves dividing the desired frequency by the frequency of the sample. This will give the rate at which the sample 
	needs to be played.
	
	Later, more samples were acquired which allowed a sample for each note. Since the uilleann pipes only have about two
	octaves of notes, it is not that many. Instead of having one sample and then pitching it to match other notes, there will be one 
	sample for each possible note on the uilleann pipes. When a key is played, the sample for that key will be played. If the key
	is not possible, then no sound is played. The drones remained unchanged. 
	
	
	\subsection{Creating Sound With Subtractive Synthesis}
	This was the original attempt which was later replaced with the above method.
	According to the presentation by T. Ó Póil \cite{uilleannSubtractive} et al, the uilleann pipe sound is mostly made up of saw and triangle waves. Through
	experimenting with a subtractive synthesiser, pulse waves also seemed to work well. For the chanter, it would use two wave 
	functions per note. When the note is played, these functions would output the waves of those frequencies for that note.
	The drones will work similarly. Since there are three drones, there will be three wave function pairs, each which will play
	the note of a drone. When toggled on, all three drones will play continuously.
	
		
		
\section {Evaluation}
	
	The major issue that occurred in the project was the lack of samples. Due to the original intent being to make
	a physical model synthesiser, it never occurred to acquire uilleann pipes to record samples. At first, this meant that 
	there was only one sample that needed to be pitch shifted. This worked fine with the same note in different octaves. Lower
	notes worked almost flawlessly while higher notes only had minimal distortion. However, while every other note does sound correct,
	they have obvious distortion and artefacting that makes the instrument not really usable. It would need some sort of interpolation
	to remove these affects. It could also be possible to change the pitch of the sample in software that already have these features.
	
	Recording high quality samples of all notes will remove the need for pitch shifting and thus removes the distortion. Of course
	this leads to more audio files and potentially a much greater file size. In the case of the uilleann pipes, it only has two octave
	of notes, so this is not actually that much. For this project, only included the sustained notes sampled. When making a proper
	sampler, recording 
	
	The only downside compared to the above method is that the instrument is limited to those notes. 
	Since there are more notes, changing octaves would be easier but there are some notes the uilleann pipes cannot play,
	no matter the octave and must be artificially created.
		
	
\printbibliography
	
		
\end{document}