\documentclass{article}
\usepackage[utf8]{inputenc}

\title{Final Project Notes}
\author{Conor Holden }


\usepackage[parfill]{parskip} % remove indent

\usepackage{hyperref}
\hypersetup{
	colorlinks=true,
	linkcolor=blue,
	filecolor=magenta,      
	urlcolor=cyan,
}

\usepackage{listings}


\begin{document}

\maketitle

\tableofcontents

\section{General Information}

	\href{http://project.cs.ucc.ie/project/567}{Project Link}
	
	\subsection{Description}
		Fiddle about with trying to develop a realistic simulation for uilleann pipes or violins, 
		or some unique instrumental sound of your own imagination.
		
	\subsection{Tools}
		\begin{itemize}
			\item \href{https://github.com/iPlug2/iPlug2}{iPlug2 GitHub} $\Rightarrow$ For Creating both plug-ins and stand-alone
			\item \href{https://github.com/iPlug2/iPlug2/wiki}{iPlug2 Wiki}
			\item \href{https://juce.com/}{Juce}  $\Rightarrow$ More mature and has more tutorials
			\item \href{https://dl.acm.org/}{ACM Digital Library}
			\item \href{https://www.youtube.com/watch?v=sxt5rxF_PdI&feature=emb_logo}{Physical Modelling}
		\end{itemize}
	
	\subsection{Videos about Iplug2 }
		Abandoned
		\begin{itemize}
			\item \href{https://www.youtube.com/watch?v=SLHGxBYeID4\&feature=youtu.be}{Oliver Larkin: Faust in iPlug 2}
			\item \href{https://www.youtube.com/watch?v=DDrgW4Qyz8Y\&feature=youtu.be}{iPlug2: Desktop Plug-in Framework Meets Web Audio Modules by Oliver Larkin}
		\end{itemize}
	 
	 
	 \subsection{Tutorials about Juce}
	 \begin{itemize}
	 	\item \href{https://docs.juce.com/master/tutorial_dsp_delay_line.html}{Juce String Model}
	 \end{itemize}

\section{Digital Signal Processing}
	\begin{itemize}
		\item \href{https://docs.juce.com/master/tutorial_dsp_introduction.html}{Juce DSP}
		\item \href{https://www.youtube.com/watch?v=HJ_-5mqUZ70}{Digital Signal Processing (DSP) Tutorial}
	\end{itemize}
	
	\subsection{Fast Fourier Transform Algorithm}
	Faster version of the Discrete Fourier transform.
	\begin{itemize}
		\item Transforms waves into its components or formula
		\item The inverse can be used to create sound waves from 
	\end{itemize}
	
	\subsection{Waves}
	\begin{itemize}
		\item Sin Wave $\Rightarrow$ std::sin (x)
		\item Saw Tooth $\Rightarrow$ map $-\pi$ -- $\pi$ to -1 -- 1 (juce::MathConstants$<$double$>$::pi)
		\item Triangle $\Rightarrow$ map $-\pi$ -- 0 to -1 -- 1 and 0 -- $\pi$ to 1 -- -1
	\end{itemize}
	
	
	\subsection{Wave Shaping}
	\begin{itemize}
		\item \href{https://docs.juce.com/master/structdsp_1_1WaveShaper.html}{dsp::WaveShaper}
	\end{itemize}
	Transforming one signal into another using a transfer function.

	
	\begin{itemize}
		\item $sin(x)$ can be converted to a square wave using signum transfer function $sgn(sin(x))$
		\item This creates a too perfect function and thus we use a hyperbolic tangent transfer function $tanh(sin(x))$
		\item To create a square, boost the singal into clipping before using the function $tanh(n*sin(x))$
	\end{itemize}
	
	\subsection{Convolution}
	\begin{itemize}
		\item \href{https://docs.juce.com/master/classdsp_1_1Convolution.html}{dsp::Convolution}
	\end{itemize}
	\textbf{Simulating the reverberation characteristics} of a certain space by using a \textbf{pre-recorded impulse} response that describes the properties of the space in question. This process allows us to apply any type of acoustic profile to an incoming signal by convolving, \textbf{essentially multiplying every sample of data against the impulse response samples} to create the combined output.
	
	
\end{document}
